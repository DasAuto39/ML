\documentclass[a4paper,12pt]{report}

\usepackage[utf8]{inputenc}
\usepackage{graphicx}
\usepackage{setspace}
\usepackage{lipsum}
\usepackage[bahasa]{babel}
\usepackage{graphicx}
\usepackage{hyperref}
\usepackage{enumitem}
\usepackage{listings}
\usepackage{xcolor}
\usepackage{tcolorbox}
\usepackage{courier}
\usepackage{fancyhdr}
\usepackage{indentfirst}


% Package TikZ untuk diagram
\usepackage{tikz}
\usepackage{pgf}
\usetikzlibrary{mindmap}
\usetikzlibrary{arrows.meta}
\usetikzlibrary{positioning}
\usetikzlibrary{shapes,arrows}
\usetikzlibrary{er}
\usetikzlibrary{calc}

% Library tambahan untuk diagram kompleks
\usetikzlibrary{trees}
\usetikzlibrary{shadows}
\usetikzlibrary{backgrounds}
\usetikzlibrary{decorations}

% Atur penomoran section agar tidak terpengaruh chapter
% Ini akan memungkinkan penomoran section dimulai dari 1 di setiap chapter
\counterwithout{section}{chapter}

% Konfigurasi hyperref
\hypersetup{
    colorlinks=true,
    linkcolor=black,
    filecolor=magenta,      
    urlcolor=cyan,
    pdftitle={Modul Praktikum Sistem Manajemen Basis Data},
    pdfpagemode=FullScreen,
}

% Konfigurasi untuk kode program
\definecolor{codegreen}{rgb}{0,0.6,0}
\definecolor{codegray}{rgb}{0.5,0.5,0.5}
\definecolor{codepurple}{rgb}{0.58,0,0.82}
\definecolor{backcolour}{rgb}{0.95,0.95,0.92}

\lstdefinestyle{mystyle}{
    backgroundcolor=\color{backcolour},
    commentstyle=\color{codegreen},
    keywordstyle=\color{magenta},
    numberstyle=\tiny\color{codegray},
    stringstyle=\color{codepurple},
    basicstyle=\ttfamily\small,
    breakatwhitespace=false,
    breaklines=true,
    captionpos=b,
    keepspaces=true,
    numbers=left,
    numbersep=5pt,
    showspaces=false,
    showstringspaces=false,
    showtabs=false,
    tabsize=2
}

\lstset{style=mystyle}

% Konfigurasi header dan footer
\pagestyle{fancy}
\fancyhf{}
\fancyhead[L]{}
\fancyhead[R]{Departemen Teknik Komputer ITS}
\fancyfoot[C]{\thepage}
\renewcommand{\headrulewidth}{0.4pt}
\renewcommand{\footrulewidth}{0.4pt}

\title{Laporan Praktikum\\
Basis Data}
\author{}
\date{}


\begin{document}

% Halaman Cover
\begin{titlepage}
    \centering

    \includegraphics[width=0.3\textwidth]{images/logo-tekkom.png}\\[1cm]

    {\Huge \textbf{Laporan Final Project Machine Learning}\\[0.3cm]}
    {\Huge \textbf{Machine Learning}}\\[1cm]
    
    {\Large \textbf{}{Prediksi Nilai Ujian Pada Sekolah Menengah ke Atas Menggunakan Model RandomForest 
    Regressor,XGboost,MLP Tensorflow,dan Linear Regresi}}\\[0.3cm]
    {\Large \textit{Muhammad Risang - 50242310xx}}\\[0.3cm]
    {\Large \textit{Akhmad Rizqullah Ridlohi - 5024231037}}\\[0.3cm]
    {\Large \textit{Abraham Napitupulu - 5024231037}}\\[0.3cm]

    
    {\large \textit{Departemen Teknik Komputer}}\\
    {\large \textit{Institut Teknologi Sepuluh Nopember}}\\
    {\large \textit{Surabaya}}\\[3cm]

    \hfill Penulis: Kelompok 6\\
    \hfill 20 Juni 2025
\end{titlepage}

\tableofcontents
\clearpage



\chapter*{Dasar Teori}
\addcontentsline{toc}{chapter}{Dasar Teori}

Pada bab ini, akan dibahas landasan teoretis yang menjadi dasar pelaksanaan proyek ini. Penjelasan mencakup konsep fundamental machine learning, alur kerja yang diterapkan, metodologi pra-pemrosesan data, serta arsitektur dan cara kerja dari setiap model yang digunakan untuk memprediksi kinerja akademik siswa.

\section{Machine Learning dan Supervised Learning}
Machine learning adalah cabang dari kecerdasan buatan (AI) yang berfokus pada pengembangan algoritma yang memungkinkan komputer untuk belajar dari data dan membuat prediksi atau keputusan tanpa diprogram secara eksplisit. Dalam proyek ini, pendekatan yang digunakan adalah Supervised Learning (Pembelajaran Terarah).

Dalam Supervised Learning, model "belajar" dari dataset yang telah memiliki label atau target yang diketahui. Model menganalisis serangkaian fitur input (variabel independen, X) dan hubungannya dengan variabel output (variabel dependen, y). Tujuan utamanya adalah untuk mempelajari fungsi pemetaan (y=f(X)) sehingga ketika data baru tanpa label diberikan, model dapat memprediksi outputnya dengan akurasi tinggi.

Tugas spesifik dalam proyek ini adalah regresi, yaitu salah satu bentuk Supervised Learning di mana variabel output (y) yang diprediksi bersifat kontinu atau numerik. Dalam konteks ini, model bertujuan untuk memprediksi nilai akhir siswa (G3) yang merupakan sebuah angka.

\section{Alur Kerja Proyek Machine Learning}
Untuk memastikan hasil yang sistematis dan dapat dipertanggungjawabkan, proyek ini mengikuti alur kerja standar dalam pengembangan model machine learning, yang terdiri dari beberapa tahapan utama:
\begin{enumerate}
\item \textbf{Eksplorasi Data (EDA):} Memahami karakteristik, distribusi, dan hubungan antar variabel dalam dataset melalui statistik deskriptif dan visualisasi data.
\item \textbf{Rekayasa Fitur (Feature Engineering):} Membuat fitur-fitur baru yang lebih informatif dari fitur yang sudah ada (misalnya, grade_avg_prev) untuk meningkatkan sinyal prediktif bagi model.
\item \textbf{Pra-pemrosesan Data (Preprocessing):} Menyiapkan dan mentransformasi data agar sesuai dengan format yang dibutuhkan oleh model machine learning.
\item \textbf{Pemodelan (Modeling):} Memilih, melatih, dan mengevaluasi beberapa algoritma yang berbeda untuk menemukan model dengan performa terbaik.
\item \textbf{Evaluasi Model:} Mengukur kinerja model terbaik menggunakan metrik kuantitatif seperti R-squared dan RMSE untuk menilai akurasi dan keandalannya.
\item \textbf{Prediksi (Inference):} Menggunakan model yang telah dilatih untuk membuat prediksi pada data baru yang belum pernah dilihat sebelumnya, seperti pada simulasi uji coba real-time.
\end{enumerate}

\section{Pra-pemrosesan Data}
Data mentah jarang sekali bisa langsung digunakan oleh model. Oleh karena itu, diperlukan beberapa teknik pra-pemrosesan untuk membersihkan dan menstrukturkan data.

\subsection{One-Hot Encoding}
Sebagian besar model machine learning hanya dapat bekerja dengan data numerik. Fitur-fitur kategorikal dalam dataset ini (seperti sex, address, higher) yang berbentuk teks perlu diubah menjadi representasi numerik. One-Hot Encoding adalah teknik yang digunakan untuk tujuan ini, di mana setiap kategori unik dalam sebuah fitur diubah menjadi kolom biner baru (bernilai 0 atau 1).

\subsection{Penskalaan Fitur (Feature Scaling)}
Fitur-fitur numerik seringkali memiliki rentang nilai yang sangat berbeda (misalnya, age (15-22) vs absences (0-93)). Jika tidak diskalakan, fitur dengan rentang nilai yang lebih besar dapat mendominasi proses pembelajaran model secara tidak adil. StandardScaler digunakan untuk mentransformasi setiap fitur sehingga memiliki rata-rata 0 dan standar deviasi 1. Langkah ini sangat krusial untuk model yang sensitif terhadap skala seperti Regresi Linear dan MLP.

\section{Model Regresi yang Digunakan}
Pemilihan model dilakukan secara strategis untuk membandingkan pendekatan dari berbagai paradigma dengan tingkat kompleksitas yang berbeda.

\subsection{Regresi Linear}
Model ini dipilih sebagai baseline atau titik awal perbandingan. Regresi Linear bekerja dengan mengasumsikan adanya hubungan linear antara fitur-fitur input dan variabel target. Ia mencoba menemukan satu formula matematis terbaik untuk memetakan input ke output. Meskipun sederhana, model ini sangat berguna untuk mengukur seberapa baik masalah dapat diselesaikan dengan pendekatan linear dan untuk menilai relevansi fitur yang dipilih.

\subsection{Random Forest}
Random Forest adalah perwakilan dari model ensemble berbasis pohon keputusan. Cara kerjanya adalah dengan membangun ratusan pohon keputusan secara independen, di mana setiap pohon dilatih pada sampel data yang sedikit berbeda. Untuk membuat prediksi, model ini mengambil rata-rata dari semua prediksi pohon individual. Pendekatan "wisdom of the crowd" ini membuatnya sangat kuat dalam menangkap interaksi fitur yang kompleks, robust terhadap noise, dan cenderung tidak overfitting, sehingga sangat cocok untuk data tabular seperti pada proyek ini.

\subsection{XGBoost}
XGBoost (eXtreme Gradient Boosting) dipilih sebagai perwakilan dari metode gradient boosting. Berbeda dari Random Forest, XGBoost membangun pohon keputusan secara sekuensial. Setiap pohon baru yang dibuat secara spesifik bertugas untuk memperbaiki kesalahan prediksi dari pohon-pohon sebelumnya. Proses belajar iteratif yang fokus pada kesalahan ini membuat XGBoost menjadi salah satu model dengan performa tertinggi di industri untuk masalah regresi dan klasifikasi pada data tabular.

\subsection{Multi-Layer Perceptron (MLP)}
MLP dipilih untuk menjelajahi pendekatan deep learning. Model ini bekerja seperti jaringan otak buatan, terdiri dari lapisan-lapisan "neuron" yang saling terhubung. MLP mampu belajar dan mengenali pola-pola yang sangat kompleks dan abstrak dari kombinasi semua fitur secara bersamaan. Model ini dipilih untuk menguji apakah arsitektur non-linear yang sangat fleksibel ini dapat menemukan pola yang terlewatkan oleh model berbasis pohon.

\section{Metrik Evaluasi Model}
Untuk mengukur dan membandingkan kinerja model secara objektif, dua metrik utama digunakan:

\begin{itemize}
\item \textbf{R-squared (R²):} Metrik ini mengukur seberapa besar persentase variasi (perbedaan) dalam variabel target (G3) yang dapat dijelaskan oleh model. Nilai R² berkisar dari 0 hingga 1, di mana nilai yang lebih tinggi menunjukkan model yang lebih baik.
\item \textbf{RMSE (Root Mean Squared Error):} Metrik ini mengukur rata-rata besaran kesalahan prediksi model dalam satuan yang sama dengan variabel target. Dalam proyek ini, RMSE menunjukkan rata-rata berapa poin nilai prediksi meleset dari nilai sebenarnya. Nilai RMSE yang lebih rendah menunjukkan model yang lebih akurat.
\end{itemize}
%Isi Dasar Teori disini   
 %Hapus jika ada modul


\chapter*{Metodologi}
\addcontentsline{toc}{chapter}{Hasil Praktikum}

\section{Model Machine Learning yang Digunakan}
Pada proyek ini, digunakan 4 model machine learning berbeda untuk dibandingkan performanya dan dipilih model yang memiliki performa terbaik. Keempat model tersebut adalah regresi linear, random forest, XGboost, dan MLP menggunakan Tensorflow.

\subsection{Regresi Linear}
Regresi linear merupakan salah satu jenis regresi, yaitu metode analisis yang dilakukan untuk mencari hubungan atau korelasi antara dua atau lebih variabel yang memiliki hubungan sebab akibat dan membuat prediksi berdasarkan hubungan tersebut. Pada regresi linear, hubungan antar variabel, yaitu variabel independen dan variabel dependen, memiliki hubungna linear (Kadam, Karhene, dan Mahindrakar. 2020). Dalam konteks machine learning, Regresi linear merupakan algoritma supervised learning yang dapat digunakan untuk melakukan prediksi nilai pada suatu label berdasarkan nilai fitur yang diberikan. Algoritma regresi linear mencari persamaan best-fit line berdasarkan fitur dan label dari dataset yang digunakan untuk training, lalu kemudian menggunakan persamaan best-fit line tersebut untuk melakukan prediksi nilai label apabila diberikan fitur dengan nilai-nilai tertentu. Model machine learning berbasis algoritma regresi linear bekerja dengan baik dalam skenario dataset yang antara fitur dan labelnya memiliki hubungan linear, namun akan memiliki performa yang buruk apabila digunakan untuk melakukan prediksi pada fitur dan label yang memiliki hubungan nonlinear.

\subsection{Random Forest}
Random forest merupakan algoritma ensemble yang bekerja dengan membuat banyak pohon keputusan (decision tree) saat proses training dan memberikan output berupa rata-rata hasil prediksi dari setiap decision tree bila digunakan untuk melakukan prediksi atau hasil mayoritas bila digunakan untuk klasifikasi. Random forest menggabungkan banyak decision tree untuk mengurangi korelasi di antara fitur pada data. Random forest menghilangkan korelasi antar pohon keputusan dengan memilih sampel secara acak dari data training, kemudian dipilih subset fitur secara acak untuk membentuk decision tree. Pengambilan sampel dan fitur secara acak ini mengurangi korelasi antara satu pohon dengan pohon lainnya, sehingga dapat mencegah terjadi overfitting dan bisa mendapatkan akurasi model yang lebih tinggi dibandingkan dengan menggunakan decision tree individual (Salman, Kalakech, dan Steiti. 2024). Pada proyek ini, digunakan model random forest dengan jumlah tree sebanyak 200 unit.

\subsection{XGBoost}
XGBoost (extreme gradient boosting) adalah algoritma ensemble decision tree yang didasarkan pada algoritma gradient boosting, yaitu algoritma yang membangun model secara aditif dengan pertimbangan minimalisasi loss function pada setiap iterasi (Bentéjac, Csörgo, dan Martínez-Muñoz. 2019). XGBoost membentuk decision tree secara sekuensial, dengan decision tree baru memperbaiki atau meningkatkan performa dari iterasi sebelumnya. Layaknya pada gradient boosting yang melakukan minimalisasi loss function untuk setiap iterasi, XGBoost menggunakan algoritma gradient descent untuk meminimalisasi loss function pada setiap pembentukan decision tree baru. Pada proyek ini, digunakan model XGBoost dengan jumlah tree sebanyak 150 unit.

\subsection{Multilayer Perceptron}
Multilayer perceptron (MLP) merupakan jaringan saraf tiruan yang setiap neuronnya menghasilkan jumlah terbobot (weighted sum) dari input neuron tersebut dan ditambahkan dengan sebuah konstan atau bias. Hasil dari proses tersebut kemudian dimasukkan ke dalam fungsi nonlinear yang disebut fungsi aktivasi (Almeida. 2020). MLP memiliki 3 komponen lapisan atau layer neuron, yaitu input layer, hidden layers, dan output layer. Input layer merupakan lapisan neuron yang menerima input berupa fitur-fitur asli yang dimiliki oleh dataset. Hidden layer merupakan lapisan yang bisa terdiri dari satu atau lebih lapisan dan menerima input dari lapisan-lapisan neuron sebelumnya. Output layer merupakan lapisan terakhir dalam jaringan saraf dan hasil dari lapisan ini adalah hasil prediksi model. Pada proyek ini, digunakan model MLP dengan 4 layer, dengan masing-masing layer (diurutkan dari input layer, hidden layer, dan output layer) memiliki 128, 64, 32, dan 1 neuron dan masing-masing neuron menggunakan fungsi aktivasi ReLU (rectified linear unit).

\section{Feature Engineering}
Pada proyek ini, dilakukan feature engineering atau rekayasa fitur untuk memperkaya informasi data pada dataset dengan tujuan meningkatkan performa model. Feature engineering didasarkan pada nilai-nilai yang memiliki korelasi kuat terhadap perubahan nilai G3, misalnya seperti nilai G1 dan G2 yang masing-masing memiliki korelasi sebesar 0,80 dan 0,90 dengan G3. Berdasarkan heatmap korelasi, nilai-nilai ini secara individual memiliki korelasi yang sangat tinggi terhadap nilai G3 dengan hubungan berbanding lurus (semakin tinggi G1 dan G2 semakin tinggi nilai G3). Kedua fitur ini dapat digabungkan untuk membuat fitur baru bernama grade\_avg\_prev, yang merupakan rata-rata nilai dari G1 dan G2, dan setelah ditambahkan ditemukan bahwa fitur baru ini juga memiliki korelasi yang tinggi dengan nilai G3, yaitu sebesar 0,89. Penambahan fitur baru melalui feature engineering membuat dataset menjadi lebih informatif, sehingga dapat meningkatkan performa dari model yang dilatih.

Melalui feature engineering berdasarkan data yang dimiliki dataset, didapatkan 5 fitur baru yang dapat digunakan oleh model, yaitu grade\_avg\_prev, total\_alcohol, study\_vs\_freetime, has\_failures, dan study\_vs\_goout. Fitur grade\_avg\_prev merupakan rata-rata dari nilai G1 dan G2. Fitur total\_alcohol merupakan total konsumsi alkohol dalam satu minggu, didapatkan dari penjumlahan nilai dalc (konsumsi alkohol di hari kerja) dan walc (konsumsi alkohol di akhir pekan). Fitur study\_vs\_freetime merupakan rasio perbandingan antara waktu belajar dan waktu senggang (keduanya dalam jam). Fitur has\_failures merupakan fitur boolean yang menyatakan bahwa siswa pernah memiliki kegagalan sebelumnya. Fitur study\_vs\_goout merupakan rasio perbandingan antara waktu belajar dengan waktu berkegiatan dan bersosialisasi di luar rumah (keduanya dalam jam).

\section{Pemilihan Model}
Pada proyek ini, digunakan 4 model machine learning yang berbeda. Keempat model ini dipilih untuk dibandingkan performanya satu sama lain untuk menemukan dengan dataset, fitur, dan label yang telah ditentukan dalam skenario proyek ini model manakah yang memiliki performa terbaik. Regresi linear dipilih sebagai baseline atau pembanding dasar, hal ini dilakukan karena regresi linear merupakan algoritma model prediktif yang paling sederhana namun memiliki potensi untuk menghasilkan prediksi yang akurat apabila diberikan training menggunakan dataset yang tepat dan digunakan dalam skenario di mana hubungan fitur dengan label sangat linear. Random forest dipilih karena random forest memiliki performa yang baik pada dataset dengan dimensionalitas tinggi, seperti dataset yang digunakan dalam proyek ini yang total memiliki 33 fitur berbeda. XGBoost dipilih karena performanya yang digadang-gadang dapat melampaui random forest, sehingga dapat menjadi pembanding bagi random forest. Multilayer perceptron (MLP) dipilih karena keserbagunaannya yang dapat digunakan di berbagai kasus, salah satunya kasus prediksi nilai. MLP juga bisa memodelkan hubungan nonlinear antara fitur dan label dengan baik, sehingga dapat dijadikan pembanding dengan regresi linear yang merupakan model yang berorientasi pada hubungan linear.

%Tugas Modul

\chapter*{Tugas Modul}
\addcontentsline{toc}{chapter}{Tugas Modul}

\begin{enumerate}
TES
\end{enumerate}


%Kesimpulan
\chapter*{Kesimpulan}
\addcontentsline{toc}{chapter}{Kesimpulan}

Setelah mendapatkan data dan hasil prediksi menunjukkan bahwa RandomForest Regressor adalah model terbaik untuk memprediksi kinerja akademik 
siswa (nilai G3) pada dataset yang digunakan. Model ini unggul dengan nilai R-squared tertinggi dan RMSE terendah, mengindikasikan akurasi 
prediksi yang baik.Meskipun model lain seperti XGBoost dan MLP TensorFlow juga menunjukkan performa yang kuat
, visualisasi dan metrik evaluasi secara konsisten menempatkan RandomForest sebagai pilihan paling optimal.Selain dari kompleksitas cara kerja  model, harus 
dipertimbangkan juga besaran dataset untuk menggunakan suatu model karena setiap model butuh data yang lebih banyak atau sedikit untuk bekerja secara optimal.
Early Data Analysis(EDA) sangatlah penting dalam menentukan fitur apa saja yang dipakai dalam model agar nilai prediksi lebih baik karena jika tidak 
dilakukan EDA ada beberapa data(Dilihat pada Heatmap) yang bisa menyebabkan performa model menjadi lebih buruk.

\chapter*{Daftar Pustaka}
\addcontentsline{toc}{chapter}{Daftar Pustaka}
Kadam, V. S., Kanhere, S., \& Mahindrakar, S. (2020). Regression techniques in machine learning \&applications: a review. Int. J. Res. Appl. Sci. Eng. Technol, 8(10), 826-830.

Salman , H.A., Kalakech , A. and Steiti, A. (trans.) (2024) “Random Forest Algorithm Overview”, Babylonian Journal of Machine Learning, 2024, pp. 69–79. doi:10.58496/BJML/2024/007.

Bentéjac, C., Csörgo, A., \& Martínez-Muñoz, G. (2019). A Comparative Analysis of XGBoost. ArXiv abs, 392.

Almeida, L. B. (2020). Multilayer perceptrons. In Handbook of Neural Computation (pp. C1-2). CRC Press.

\begin{enumerate}
    \item Al-Barrak, M. A., \& Al-Razgan, M. {(2016)}. Predicting Students Final GPA Using Decision Trees: A Case Study. \textit{International Journal of Information and Education Technology, 6}{(7)}, 528-533. doi:10.7763{/}IJIET.2016.V6.745.
    \item  Almeida, L. B. (2020). Multilayer perceptrons. In Handbook of Neural Computation (pp. C1-2). CRC Press.
    \item Cortez, P. (2008). Student Performance [Dataset]. UCI Machine Learning Repository. https://doi.org/10.24432/C5TG7T.
    \item Bentéjac, C., Csörgo, A., \& Martínez-Muñoz, G. (2019). A Comparative Analysis of XGBoost. ArXiv abs, 392.
    \item Kadam, V. S., Kanhere, S., \& Mahindrakar, S. (2020). Regression techniques in machine learning \&applications: a review. Int. J. Res. Appl. Sci. Eng. Technol, 8(10), 826-830.
    \item Salman , H.A., Kalakech , A. and Steiti, A. (trans.) (2024) “Random Forest Algorithm Overview”, Babylonian Journal of Machine Learning, 2024, pp. 69–79. doi:10.58496/BJML/2024/007.

    \item GeeksforGeeks. (n.d.). \textit{Multi-Layer Perceptron Learning in TensorFlow}. Diakses dari https://www.geeksforgeeks.org/deep-learning/multi-layer-perceptron-learning-in-tensorflow/
    
    \item GeeksforGeeks. (n.d.). \textit{XGBoost - Introduction to Boosting Algorithm}. Diakses dari https://www.geeksforgeeks.org/machine-learning/xgboost/
    
    \item GeeksforGeeks. (n.d.). \textit{Random Forest Regression in Python}. Diakses dari https://www.geeksforgeeks.org/machine-learning/random-forest-regression-in-python/
\end{enumerate}


\end{document}
