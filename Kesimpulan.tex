\chapter*{Kesimpulan}
\addcontentsline{toc}{chapter}{Kesimpulan}

Setelah mendapatkan data dan hasil prediksi menunjukkan bahwa RandomForest Regressor adalah model terbaik untuk memprediksi kinerja akademik 
siswa (nilai G3) pada dataset yang digunakan. Model ini unggul dengan nilai R-squared tertinggi dan RMSE terendah, mengindikasikan akurasi 
prediksi yang baik.Meskipun model lain seperti XGBoost dan MLP TensorFlow juga menunjukkan performa yang kuat
, visualisasi dan metrik evaluasi secara konsisten menempatkan RandomForest sebagai pilihan paling optimal.Selain dari kompleksitas cara kerja  model, harus 
dipertimbangkan juga besaran dataset untuk menggunakan suatu model karena setiap model butuh data yang lebih banyak atau sedikit untuk bekerja secara optimal.
Early Data Analysis(EDA) sangatlah penting dalam menentukan fitur apa saja yang dipakai dalam model agar nilai prediksi lebih baik karena jika tidak 
dilakukan EDA ada beberapa data(Dilihat pada Heatmap) yang bisa menyebabkan performa model menjadi lebih buruk.

\chapter*{Daftar Pustaka}
\addcontentsline{toc}{chapter}{Daftar Pustaka}
\begin{enumerate}
    \item Al-Barrak, M. A., \& Al-Razgan, M. {(2016)}. Predicting Students Final GPA Using Decision Trees: A Case Study. \textit{International Journal of Information and Education Technology, 6}{(7)}, 528-533. doi:10.7763{/}IJIET.2016.V6.745.
    \item  Almeida, L. B. (2020). Multilayer perceptrons. In Handbook of Neural Computation (pp. C1-2). CRC Press.
    \item Cortez, P. (2008). Student Performance [Dataset]. UCI Machine Learning Repository. https://doi.org/10.24432/C5TG7T.
    \item Bentéjac, C., Csörgo, A., \& Martínez-Muñoz, G. (2019). A Comparative Analysis of XGBoost. ArXiv abs, 392.
    \item Kadam, V. S., Kanhere, S., \& Mahindrakar, S. (2020). Regression techniques in machine learning \&applications: a review. Int. J. Res. Appl. Sci. Eng. Technol, 8(10), 826-830.
    \item Salman , H.A., Kalakech , A. and Steiti, A. (trans.) (2024) “Random Forest Algorithm Overview”, Babylonian Journal of Machine Learning, 2024, pp. 69–79. doi:10.58496/BJML/2024/007.

    \item GeeksforGeeks. (n.d.). \textit{Multi-Layer Perceptron Learning in TensorFlow}. Diakses dari https://www.geeksforgeeks.org/deep-learning/multi-layer-perceptron-learning-in-tensorflow/
    
    \item GeeksforGeeks. (n.d.). \textit{XGBoost - Introduction to Boosting Algorithm}. Diakses dari https://www.geeksforgeeks.org/machine-learning/xgboost/
    
    \item GeeksforGeeks. (n.d.). \textit{Random Forest Regression in Python}. Diakses dari https://www.geeksforgeeks.org/machine-learning/random-forest-regression-in-python/
\end{enumerate}
