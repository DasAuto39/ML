\chapter*{Metodologi}
\addcontentsline{toc}{chapter}{Hasil Praktikum}

\section{Model Machine Learning yang Digunakan}
Pada proyek ini, digunakan 4 model machine learning berbeda untuk dibandingkan performanya dan dipilih model yang memiliki performa terbaik. Keempat model tersebut adalah regresi linear, random forest, XGboost, dan MLP menggunakan Tensorflow.

\subsection{Regresi Linear}
Regresi linear merupakan salah satu jenis regresi, yaitu metode analisis yang dilakukan untuk mencari hubungan atau korelasi antara dua atau lebih variabel yang memiliki hubungan sebab akibat dan membuat prediksi berdasarkan hubungan tersebut. Pada regresi linear, hubungan antar variabel, yaitu variabel independen dan variabel dependen, memiliki hubungna linear (Kadam, Karhene, dan Mahindrakar. 2020). Dalam konteks machine learning, Regresi linear merupakan algoritma supervised learning yang dapat digunakan untuk melakukan prediksi nilai pada suatu label berdasarkan nilai fitur yang diberikan. Algoritma regresi linear mencari persamaan best-fit line berdasarkan fitur dan label dari dataset yang digunakan untuk training, lalu kemudian menggunakan persamaan best-fit line tersebut untuk melakukan prediksi nilai label apabila diberikan fitur dengan nilai-nilai tertentu. Model machine learning berbasis algoritma regresi linear bekerja dengan baik dalam skenario dataset yang antara fitur dan labelnya memiliki hubungan linear, namun akan memiliki performa yang buruk apabila digunakan untuk melakukan prediksi pada fitur dan label yang memiliki hubungan nonlinear.

\subsection{Random Forest}
Random forest merupakan algoritma ensemble yang bekerja dengan membuat banyak pohon keputusan (decision tree) saat proses training dan memberikan output berupa rata-rata hasil prediksi dari setiap decision tree bila digunakan untuk melakukan prediksi atau hasil mayoritas bila digunakan untuk klasifikasi. Random forest menggabungkan banyak decision tree untuk mengurangi korelasi di antara fitur pada data. Random forest menghilangkan korelasi antar pohon keputusan dengan memilih sampel secara acak dari data training, kemudian dipilih subset fitur secara acak untuk membentuk decision tree. Pengambilan sampel dan fitur secara acak ini mengurangi korelasi antara satu pohon dengan pohon lainnya, sehingga dapat mencegah terjadi overfitting dan bisa mendapatkan akurasi model yang lebih tinggi dibandingkan dengan menggunakan decision tree individual (Salman, Kalakech, dan Steiti. 2024). Pada proyek ini, digunakan model random forest dengan jumlah tree sebanyak 200 unit.

\subsection{XGBoost}
XGBoost (extreme gradient boosting) adalah algoritma ensemble decision tree yang didasarkan pada algoritma gradient boosting, yaitu algoritma yang membangun model secara aditif dengan pertimbangan minimalisasi loss function pada setiap iterasi (Bentéjac, Csörgo, dan Martínez-Muñoz. 2019). XGBoost membentuk decision tree secara sekuensial, dengan decision tree baru memperbaiki atau meningkatkan performa dari iterasi sebelumnya. Layaknya pada gradient boosting yang melakukan minimalisasi loss function untuk setiap iterasi, XGBoost menggunakan algoritma gradient descent untuk meminimalisasi loss function pada setiap pembentukan decision tree baru. Pada proyek ini, digunakan model XGBoost dengan jumlah tree sebanyak 150 unit.

\subsection{Multilayer Perceptron}
Multilayer perceptron (MLP) merupakan jaringan saraf tiruan yang setiap neuronnya menghasilkan jumlah terbobot (weighted sum) dari input neuron tersebut dan ditambahkan dengan sebuah konstan atau bias. Hasil dari proses tersebut kemudian dimasukkan ke dalam fungsi nonlinear yang disebut fungsi aktivasi (Almeida. 2020). MLP memiliki 3 komponen lapisan atau layer neuron, yaitu input layer, hidden layers, dan output layer. Input layer merupakan lapisan neuron yang menerima input berupa fitur-fitur asli yang dimiliki oleh dataset. Hidden layer merupakan lapisan yang bisa terdiri dari satu atau lebih lapisan dan menerima input dari lapisan-lapisan neuron sebelumnya. Output layer merupakan lapisan terakhir dalam jaringan saraf dan hasil dari lapisan ini adalah hasil prediksi model. Pada proyek ini, digunakan model MLP dengan 4 layer, dengan masing-masing layer (diurutkan dari input layer, hidden layer, dan output layer) memiliki 128, 64, 32, dan 1 neuron dan masing-masing neuron menggunakan fungsi aktivasi ReLU (rectified linear unit).

\section{Feature Engineering}
Pada proyek ini, dilakukan feature engineering atau rekayasa fitur untuk memperkaya informasi data pada dataset dengan tujuan meningkatkan performa model. Feature engineering didasarkan pada nilai-nilai yang memiliki korelasi kuat terhadap perubahan nilai G3, misalnya seperti nilai G1 dan G2 yang masing-masing memiliki korelasi sebesar 0,80 dan 0,90 dengan G3. Berdasarkan heatmap korelasi, nilai-nilai ini secara individual memiliki korelasi yang sangat tinggi terhadap nilai G3 dengan hubungan berbanding lurus (semakin tinggi G1 dan G2 semakin tinggi nilai G3). Kedua fitur ini dapat digabungkan untuk membuat fitur baru bernama grade\_avg\_prev, yang merupakan rata-rata nilai dari G1 dan G2, dan setelah ditambahkan ditemukan bahwa fitur baru ini juga memiliki korelasi yang tinggi dengan nilai G3, yaitu sebesar 0,89. Penambahan fitur baru melalui feature engineering membuat dataset menjadi lebih informatif, sehingga dapat meningkatkan performa dari model yang dilatih.

Melalui feature engineering berdasarkan data yang dimiliki dataset, didapatkan 5 fitur baru yang dapat digunakan oleh model, yaitu grade\_avg\_prev, total\_alcohol, study\_vs\_freetime, has\_failures, dan study\_vs\_goout. Fitur grade\_avg\_prev merupakan rata-rata dari nilai G1 dan G2. Fitur total\_alcohol merupakan total konsumsi alkohol dalam satu minggu, didapatkan dari penjumlahan nilai dalc (konsumsi alkohol di hari kerja) dan walc (konsumsi alkohol di akhir pekan). Fitur study\_vs\_freetime merupakan rasio perbandingan antara waktu belajar dan waktu senggang (keduanya dalam jam). Fitur has\_failures merupakan fitur boolean yang menyatakan bahwa siswa pernah memiliki kegagalan sebelumnya. Fitur study\_vs\_goout merupakan rasio perbandingan antara waktu belajar dengan waktu berkegiatan dan bersosialisasi di luar rumah (keduanya dalam jam).

\section{Pemilihan Model}
Pada proyek ini, digunakan 4 model machine learning yang berbeda. Keempat model ini dipilih untuk dibandingkan performanya satu sama lain untuk menemukan dengan dataset, fitur, dan label yang telah ditentukan dalam skenario proyek ini model manakah yang memiliki performa terbaik. Regresi linear dipilih sebagai baseline atau pembanding dasar, hal ini dilakukan karena regresi linear merupakan algoritma model prediktif yang paling sederhana namun memiliki potensi untuk menghasilkan prediksi yang akurat apabila diberikan training menggunakan dataset yang tepat dan digunakan dalam skenario di mana hubungan fitur dengan label sangat linear. Random forest dipilih karena random forest memiliki performa yang baik pada dataset dengan dimensionalitas tinggi, seperti dataset yang digunakan dalam proyek ini yang total memiliki 33 fitur berbeda. XGBoost dipilih karena performanya yang digadang-gadang dapat melampaui random forest, sehingga dapat menjadi pembanding bagi random forest. Multilayer perceptron (MLP) dipilih karena keserbagunaannya yang dapat digunakan di berbagai kasus, salah satunya kasus prediksi nilai. MLP juga bisa memodelkan hubungan nonlinear antara fitur dan label dengan baik, sehingga dapat dijadikan pembanding dengan regresi linear yang merupakan model yang berorientasi pada hubungan linear.